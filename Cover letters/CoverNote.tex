%\documentclass{oxmathletter} % if using pdflatex
\documentclass[xelatex,nofoundrysterling]{oxphysicsletter} % if using xelatex

%following options can be combined if desired and used with xelatex option
%\documentclass[cmrbody]{oxmathletter} % use computer modern roman font
%\documentclass[nofoundrysterling]{oxmathletter} % don't have the oxford font

\usepackage{amsmath}
\usepackage{amsfonts}
\usepackage{amsxtra}
\usepackage{graphicx}
\usepackage{cleveref}
\usepackage{siunitx}
\usepackage{enumitem,doi} 
\pretolerance=10000
\tolerance=2000 
\emergencystretch=10pt

\signature{\vspace{-1em}Dr Jack J J J Miller, DPhil, MPhys, AFHEA, MIPEM, MInstP.} % your name as you want it to appear as signature

% This document class should only be used for official departmental business
% As such the option defaults are the official departmental general contact
% details. Defaults should only be overridden with your personal official 
% departmental contact details

%%%%%%%%%%%%%%%%%%%%%%%%%%%%%%%%%%%%%%%%%%%%%%%%%%%%%%%%%%%%%%%%%%%%%%%%%%%%
%optional commands to override defaults
%do not include these commands if you wish to stick with the defaults

\position{\vspace{-0.25em}\small Novo Nordisk Postdoctoral Fellow,\\Department of Physiology, Anatomy and Genetics\\Department of Physics\\\vspace{0.75em}Junior Research Fellow in the Medical Sciences\\Wadham College in the University of Oxford \\\vspace{0.75em}Stipendiary Lecturer in Physics,\\ St. Hugh's College in the University of Oxford} % your role/position, default none shown

\phone{272560}   % your direct phone number (area code is added automatically)
                 % default none shown

\fax{270515}     % fax number (area code is added automatically)
                 % defaults to St Giles reception fax

\econtact{jack.miller}  % your official (long format) maths e-mail address 
                         % without @maths.ox.ac.uk 
                         % this will produce your official email address
                         % and the URL of your official web profile page

%\refcode{MIIT/2013/17A} % optional reference code to specify on letter
                        % default none shown

%\researchgroup{Oxford Centre for Mundane Mathematics} % specify a research 
                                                       % group name to show 
                                             %above "Mathematical Institute"

%end of optional commands
%%%%%%%%%%%%%%%%%%%%%%%%%%%%%%%%%%%%%%%%%%%%%%%%%%%%%%%%%%%%%%%%%%%%%%%%%%%%
\date{29th February, 2020}
\setlength{\footnotesep}{2mm}
\begin{document}
\begin{letter}{\vspace*{-0em}

Professor Andrew Flewitt \& Committee \\
University of Cambridge }
%Prof. Matt A Bernstein, PhD \\  
%Professor of Medical Physics, Mayo Clinic\\
%Rochester, Minnesota\\
%USA}
    \subject{Re: The Gianna Angelopoulos University Lectureship in Medical Therapeutics}
% optional subject line
    \opening{\vspace*{-01em}Dear Committee,}  % initial greeting/opening

    I am delighted to submit in this document three publications, as requested, which showcase a brief portfolio of my work. 

    Please note that many of the technologies I have developed have been applied in conjunction with medics and clinical research colleagues to produce several significant and impactful pieces of work. Rather than present their applications, I have deliberately elected to provide more technical, detailed papers that include more detail about three methods I have developed.  

    Please find attached: 
    \begin{enumerate} 
        \item \textbf{Miller, J J.}, Lau, A Z., Teh., I., Schneider, J., Kinchecsh, P., Smart, S., Ball, V., Sibson, N R., Tyler, D J. (2015) Robust, high resolution three-dimensional hyperpolarised metabolic imaging of the healthy rat heart at 7 T. Magnetic Resonance in Medicine, \doi{10.1002/mrm.25730}.
    \end{enumerate} 

    This work, one of seven published during my DPhil [PhD], describes a novel pulse sequence to image the metabolism of hyperpolarised [1-\textsuperscript{13}C]pyruvate at, for the time, world-leading spatiotemporal resolution. It has been recognised as a key paper in the field, and, as of February 2020, has been cited 40 times. For context, there are approximately 13 sites worldwide with the ability to undertake human hyperpolarised experiments. It is the basis of the majority of other imaging strategies taken by sites worldwide, and my specific implementation was used in a large number of other high-impact pieces of work. 

    \begin{enumerate}[resume] 
    \item \textbf{Miller, J J.}, Lau, A Z., Nielsen, P M., McMullen-Klein, G., Lewis, A J.,  Jespersen, N R., Ball, V., Gallagher, F A., Carr, C A., Laustsen, C., B\o{}tker, H E., Tyler, D J., Schroeder M. A. (2017): Hyperpolarized [1,4-$^{13}$C$_2$]Fumarate Enables Magnetic Resonance-Based Imaging of Myocardial Necrosis. Journal of the American College of Cardiology Cardiovascular Imaging, \doi{10.1016/j.jcmg.2017.09.020}. \emph{This article was also the subject of an editorial in JACC: Cardiovascular Imaging, c.f. JACC: Cardiovascular Imaging, \doi{10.1016/j.jcmg.2017.10.015}, which is also attached.}
    \end{enumerate}  

    This work, performed in collaboration with Danish colleagues, describes a novel molecule and application, performed during my EPSRC Doctoral Prize Fellowship, namely the use of hyperpolarized [1,4-\textsuperscript{13}C]fumarate to quantify and probe myocardial necrosis directly. The basic idea is that the transport of fumarate into the cell is slow compared to its visible duration in the hyperpolarised experiment (i.e. its $T_1$) and therefore the downstream metabolic product, malate, is only visible if the cell has been lysed. This work has been cited as part of an evidence package used to provide justification for driving fumarate into clinical trials by GE healthcare, among others.

    \begin{enumerate}[resume] 
\item \textbf{Miller, J J.}, Grist, J T., Serres, S S., Fisher, K., Larkin, J., Lau, A Z., Ray, K.,  Hansen, E S-S., Tougaard, R., Nielsen, P M., Lindhardt, J., Sibson, N R., Laustsen, C L., Gallagher, F A., Tyler, D J. Preclinical Hyperpolarised MRI with $^{13}$C Pyruvate and Transport Across the Blood-Brain-Barrier. Nature Scientific Reports (2018), 8 (1), 15082, \doi{10.1038/s41598-018-33363-5}. 
    \end{enumerate} 

    This work, performed during my Novo Nordisk fellowship and in collaboration with researchers from Cambridge and Denmark, demonstrates conclusively in experiments on a variety of living systems that pre-clinical experiments looking in the brain with hyperpolarised pyruvate had been potentially confused by a combination of physical and biological effects. Specifically, that transport of hyperpolarised pyruvate across the blood-brain-barrier may be limited by transport, and that other reports depicting cerebral metabolism may have been confounded by partial volume effects, with the point-spread-function of the imaging sequences used by others being substantially worse than my previous work.  When combined with a very bright signal present in the blood pool, this work forms conclusive evidence that measurements made by others were not necessarily what they appeared to be.

    This work, although new, has been discussed extensively by the community, cited as evidence for moving forward into human trials of hyperpolarised brain metabolism, and confirmed independently by others in  the field. It also proposes a novel method  for the  imaging of hyperpolarised ethyl-pyruvate, and introduces a number of technical advances that  have been further utilised by the community at large. 

I look forward to meeting you, and thank you again for the time you have spent in assessing my application, 

\closing{\vspace{-2em}Yours sincerely,\vspace*{-1em}} % closing phrase
%\encl{CV, including list of publications}
\end{letter}
\end{document}
